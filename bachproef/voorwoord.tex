%%=============================================================================
%% Voorwoord
%%=============================================================================

\chapter*{\IfLanguageName{dutch}{Woord vooraf}{Preface}}%
\label{ch:voorwoord}

%% TODO:
%% Het voorwoord is het enige deel van de bachelorproef waar je vanuit je
%% eigen standpunt (``ik-vorm'') mag schrijven. Je kan hier bv. motiveren
%% waarom jij het onderwerp wil bespreken.
%% Vergeet ook niet te bedanken wie je geholpen/gesteund/... heeft
Dit onderzoek is geschreven in samenwerking met het bedrijf Aucxis en vanuit een persoonlijke interesse die dateert van de middelbare school, toen ik ook een eindwerk had gemaakt over Bluetooth. Ik was voor ik aan dit onderzoek begon niet bekend met Bluetooth Low Energy en heb veel bijgeleerd door deze scriptie te schrijven. Ik heb de combinatie van hardware en software altijd interessant gevonden dus daarom was ik gemotiveerd om dit onderzoek uit te voeren.\\

Deze bachelorproef werd geschreven in kader van het voltooien van mijn opleiding Toegepaste Informatica met als afstudeerrichting Mobile and Enterprise Developer.\\

Deze scriptie zou niet tot stand gekomen zijn indien ik geen hulp had gekregen van bepaalde personen die ik hier graag zou willen bedanken. Jason Scrivens (Aucxis) en George Liekens, mijn co-promotor, die mij samen voorgesteld hebben om over dit onderwerp te schrijven. Leen Vuyge, mijn promotor, voor steeds klaar te staan om, te bellen, mijn vragen te beantwoorden en feedback te geven.\\ 

Ik hoop dat degene die dit onderzoek leest, iets zal bijleren over alle verschillende technologieën die aan bod gekomen zijn. Net zoals ik heb bijgeleerd tijdens het schrijven. 