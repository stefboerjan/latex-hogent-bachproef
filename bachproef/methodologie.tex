%%=============================================================================
%% Methodologie
%%=============================================================================

\chapter{\IfLanguageName{dutch}{Methodologie}{Methodology}}%
\label{ch:methodologie}

%% TODO: Hoe ben je te werk gegaan? Verdeel je onderzoek in grote fasen, en
%% licht in elke fase toe welke stappen je gevolgd hebt. Verantwoord waarom je
%% op deze manier te werk gegaan bent. Je moet kunnen aantonen dat je de best
%% mogelijke manier toegepast hebt om een antwoord te vinden op de
%% onderzoeksvraag.

Zoals eerder vermeld zal dit onderzoek zich focussen op het voorstel dat Bluetooth Low Energy een optimale oplossing is voor asset tracking in de bouwindustrie. Tegenwoordig worden daar vooral andere technologieën voor gebruikt. BLE zit daar reeds wel al tussen maar in veel mindere mate. In dit gedeelte van de scripte zal er een vergelijking gemaakt worden tussen de verschillende technologieën die vandaag de dag vooral gebruikt worden. Er zal een grondige beschrijving gebeuren van alle voor en nadelen van BLE. Ook bestaat dit deel uit een toelichting van alle geschikte protocols, software en hardware. Op basis hiervan er een experimenteel onderzoek plaatsvinden aan de hand van een zelfontwikkelde Android-applicatie.

%De eerste fase is een introductie over de pro-blematiek. Dit wordt gerealiseerd door een gron-dige studie van vakliteratuur, zoals wetenschap-pelijke teksten of blogs. Hieruit volgt een tekstdie alle vereisten aanhaalt voor een optimale op-lossing. De geschatte duurtijd van deze fase be-draagt twee weken.
\subsection{Voordelen van BLE asset tracking}

De tweede fase houdt een analyse in van we-tenschappelijke teksten, met als resultaat een uit-gebreide tekst over de voordelen van BLE assettracking, ten opzichte van technologieën die van-daag de dag gebruikt worden. Hiervoor is eenweek genoeg.
\subsection{Nadelen van BLE asset tracking}

De derde fase is opnieuw een beschrijving,maar dan over de valkuilen van BLE asset trac-king.  Deze fase van het onderzoek brengt allemogelijke tekortkomingen in kaart.  Ook voordeze fase is een week voldoende.
\subsection{Toelichting van protocols, hardware en software}

De vierde fase bestaat uit het toelichten vande meest geschikte protocols, compatibiliteit vansoftware en hardware ,zoals beacons en gateways.Hierbij wordt er gezocht naar de meest geschiktehardware waarbij er nauwkeurig afgewogen moetw orden tussen kosten en functionaliteit. Vervol-gens wordt er een veldonderzoek uitgevoerd om,naargelang de vereisten, de keuze te maken opwelke manier de app ontwikkeld zal worden. Totslot zullen alle beschikbare protocols in kaart ge-bracht worden. De geschatte duurtijd van dezefase bedraagt drie weken.

\subsection{Toelichting app}

