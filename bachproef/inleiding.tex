%%=============================================================================
%% Inleiding
%%=============================================================================

\chapter{\IfLanguageName{dutch}{Inleiding}{Introduction}}%
\label{ch:inleiding}

%De inleiding moet de lezer net genoeg informatie verschaffen om het onderwerp te begrijpen en in te zien waarom de onderzoeksvraag de moeite waard is om te onderzoeken. In de inleiding ga je literatuurverwijzingen beperken, zodat de tekst vlot leesbaar blijft. Je kan de inleiding verder onderverdelen in secties als dit de tekst verduidelijkt. Zaken die aan bod kunnen komen in de inleiding~\autocite{Pollefliet2011}:
%
%\begin{itemize}
%  \item context, achtergrond
%  \item afbakenen van het onderwerp
%  \item verantwoording van het onderwerp, methodologie
%  \item probleemstelling
%  \item onderzoeksdoelstelling
%  \item onderzoeksvraag
%  \item \ldots
%\end{itemize}

Asset tracking is een veel voorkomend fenomeen wanneer een bedrijf de locatie van zijn fysieke activa wil bijhouden. Dit wordt gedaan in diverse sectoren en in de bouwindustrie is dit niet anders. Asset tracking in de bouwsector wordt reeds gedaan aan de hand van global positioning system (GPS), ultra-windband (UWB) en radio-frequency identification (RFID). Deze technologieën worden gebruikt om de locatie van materiaal en personeel op een complexe bouwwerf te bepalen en bij te houden. Een recente technologie genaamd Bluetooth Low Energy (BLE), of wordt soms ook Bluetooth Smart genoemd is een opkomende concurrent voor deze reeds gebruikte technologieën. \\

Bluetooth Low Energy is een draadloos personal area netwerk (PAN) dat werd ontwikkeld met Internet of Things (IoT) in gedachten. IoT applicaties hebben weinig resources en daardoor is BLE net ontwikkeld. Het wordt vandaag de dag virtueel overal gebruikt, gaande van fitness trackers en smart appliences tot location tracking en contact tracing. Use cases waar geen grote hoeveelheid data moet uitgewisseld worden of waar niet veel vermogen beschikbaar is. Vergeleken met het klassieke bluetooth is BLE bedoeld om het energieverbruik en de kosten aanzienlijk te doen verminderen, terwijl het communicatiebereik vergelijkbaar blijft. Bluetooth Low Energy wordt door bijna elk operating system ondersteund zoals Windows, macOS, Linux,  Android en iOS.

waarom kan dit gebruikt wordeni n de bouw

hoe kan dit bijdragen?






\section{\IfLanguageName{dutch}{Probleemstelling}{Problem Statement}}%
\label{sec:probleemstelling}

%Uit je probleemstelling moet duidelijk zijn dat je onderzoek een meerwaarde heeft voor een concrete doelgroep. De doelgroep moet goed gedefinieerd en afgelijnd zijn. Doelgroepen als ``bedrijven,'' ``KMO's'', systeembeheerders, enz.~zijn nog te vaag. Als je een lijstje kan maken van de personen/organisaties die een meerwaarde zullen vinden in deze bachelorproef (dit is eigenlijk je steekproefkader), dan is dat een indicatie dat de doelgroep goed gedefinieerd is. Dit kan een enkel bedrijf zijn of zelfs één persoon (je co-promotor/opdrachtgever).

In de bouwindustrie wordt er aan asset tracking gedaan a.d.h.v. al reeds vermelde technologieën. Bluetooth Low Energy hoort hier in de grote meerderheid van de bedrijven nog niet bij ondanks het potentieel en geschiktheid. Kostreductie is een van de belangrijkste processen binnenin een succesvol bedrijf en hier kan BLE optimaal aan bijdragen  in het kader van asset tracking door de goedkope hardware en lange gebruiksduur.\\

Bluetooth Low Energy wordt al reeds gebruikt voor indoor traceren. Op het vlak van oppervlakte is een bouwwerf niet erg verschillend met een groot magazijn of bedrijfsgebouw.
Werfleiders en andere verantwoordelijken kunnen aan de hand van Bluetooth Low Energy op dezelfde manier hun werkmateriaal, personeel en machines traceren, maar dan veel goedkoper, accurater en mogelijks efficiënter. \\

Het bedrijf Aucxis, te Stekene liet mij hier onderzoek over doen en ook zij zullen hier meerwaarde in vinden aangezien ook zij Bluetooth Low Energy in hun asset tracking oplossingen willen integreren.

\section{\IfLanguageName{dutch}{Onderzoeksvraag}{Research question}}%
\label{sec:onderzoeksvraag}

%Wees zo concreet mogelijk bij het formuleren van je onderzoeksvraag. Een onderzoeksvraag is trouwens iets waar nog niemand op dit moment een antwoord heeft (voor zover je kan nagaan). Het opzoeken van bestaande informatie (bv. ``welke tools bestaan er voor deze toepassing?'') is dus geen onderzoeksvraag. Je kan de onderzoeksvraag verder specifiëren in deelvragen. Bv.~als je onderzoek gaat over performantiemetingen, dan 

Om deze problematiek uit te zoeken is een onderzoeksvraag opgesteld met bijhorende deelvragen. Hier zal antwoord op proberen gegeven worden door een literatuurstudie en een experimentele uitvoering.

\begin{itemize}
    \item Kan Bluetooth Low Energy asset tracking op een bouwwerf optimaliseren?
    \begin{itemize}
        \item Deelvragen...
    \end{itemize}
\end{itemize}

\section{\IfLanguageName{dutch}{Onderzoeksdoelstelling}{Research objective}}%
\label{sec:onderzoeksdoelstelling}

%Wat is het beoogde resultaat van je bachelorproef? Wat zijn de criteria voor succes? Beschrijf die zo concreet mogelijk. Gaat het bv.\ om een proof-of-concept, een prototype, een verslag met aanbevelingen, een vergelijkende studie, enz.

Het doel van deze scriptie is om te bewijzen dat al dan niet Bluetooth Low Energy een optimale(re) technologie is om assets op een bouwwerf te tracken. De voornaamste criteria zullen kosten, beveiliging, traceernauwkeurigheid en hoe makkelijk de installatie is. Dit zal vergeleken worden met de technologieën die vandaag de dag gebruikt worden door de meerderheid van de bouwbedrijven.

\section{\IfLanguageName{dutch}{Opzet van deze bachelorproef}{Structure of this bachelor thesis}}%
\label{sec:opzet-bachelorproef}

% Het is gebruikelijk aan het einde van de inleiding een overzicht te
% geven van de opbouw van de rest van de tekst. Deze sectie bevat al een aanzet
% die je kan aanvullen/aanpassen in functie van je eigen tekst.

De rest van deze bachelorproef is als volgt opgebouwd:

In Hoofdstuk~\ref{ch:stand-van-zaken} wordt een overzicht gegeven van de stand van zaken binnen het onderzoeksdomein, op basis van een literatuurstudie.

In Hoofdstuk~\ref{ch:methodologie} wordt de methodologie toegelicht en worden de gebruikte onderzoekstechnieken besproken om een antwoord te kunnen formuleren op de onderzoeksvragen.

% TODO: Vul hier aan voor je eigen hoofstukken, één of twee zinnen per hoofdstuk

In Hoofdstuk~\ref{ch:conclusie}, tenslotte, wordt de conclusie gegeven en een antwoord geformuleerd op de onderzoeksvragen. Daarbij wordt ook een aanzet gegeven voor toekomstig onderzoek binnen dit domein.