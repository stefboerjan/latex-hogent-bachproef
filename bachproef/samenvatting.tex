%%=============================================================================
%% Samenvatting
%%=============================================================================

% TODO: De "abstract" of samenvatting is een kernachtige (~ 1 blz. voor een
% thesis) synthese van het document.
%
% Een goede abstract biedt een kernachtig antwoord op volgende vragen:
%
% 1. Waarover gaat de bachelorproef?
% 2. Waarom heb je er over geschreven?
% 3. Hoe heb je het onderzoek uitgevoerd?
% 4. Wat waren de resultaten? Wat blijkt uit je onderzoek?
% 5. Wat betekenen je resultaten? Wat is de relevantie voor het werkveld?
%
% Daarom bestaat een abstract uit volgende componenten:
%
% - inleiding + kaderen thema
% - probleemstelling
% - (centrale) onderzoeksvraag
% - onderzoeksdoelstelling
% - methodologie
% - resultaten (beperk tot de belangrijkste, relevant voor de onderzoeksvraag)
% - conclusies, aanbevelingen, beperkingen
%
% LET OP! Een samenvatting is GEEN voorwoord!

%%---------- Nederlandse samenvatting -----------------------------------------
%
% TODO: Als je je bachelorproef in het Engels schrijft, moet je eerst een
% Nederlandse samenvatting invoegen. Haal daarvoor onderstaande code uit
% commentaar.
% Wie zijn bachelorproef in het Nederlands schrijft, kan dit negeren, de inhoud
% wordt niet in het document ingevoegd.

\IfLanguageName{english}{%
\selectlanguage{dutch}
\chapter*{Samenvatting}
\lipsum[1-4]
\selectlanguage{english}
}{}

%%---------- Samenvatting -----------------------------------------------------
% De samenvatting in de hoofdtaal van het document

\chapter*{\IfLanguageName{dutch}{Samenvatting}{Abstract}}
Asset tracking is een fenomeen dat voorvalt indien een bedrijf zijn fysieke activa wil traceren. Dit wordt toegepast in verschillende sectoren en bij de bouwindustrie is dit niet anders. Het is niet simpel om al het personeel, voertuigen, machines en werkmateriaal snel te vinden op een complexe bouwwerf zonder het gebruik van asset tracking. Dit wordt reeds gedaan door gebruik te maken van verschillende technologieën. Global Positioning System (GPS), Radio Frequentie Identifier (RFID), Ultra-wideband (UWB) en Bluetooth Low Energy (BLE). Bluetooth Low Energy is in de grote meerderheid van bouwbedrijven nog niet zo populair als asset tracking technologie. In dit onderzoek zal nagegaan worden of asset tracking op een bouwwerf kan geoptimaliseerd worden door gebruik te maken van Bluetooth Low Energy. In deze scriptie vind u een inleiding tot het onderwerp en een beschrijving en vergelijking van de meest frequent voorkomende asset tracking technologieën, ook buiten de bouwindustrie. Hierdoor worden geen technologieën buitengesloten. Bij deze vergelijkende analyse wordt vooral gekeken naar zaken als werkgebied, traceernauwkeurigheid, kosten, energieverbruik en functionaliteit. Uit dit onderzoek blijkt dat er bepaalde technologieën zijn die totaal niet geschikt zijn zoals barcode, QR code en RFID indien de technologie smartphone compatibel moet zijn, maar dat het voor de andere technologieën afwegen is volgens prioriteiten. Aangezien de overige technologieën elk exceptionele functionaliteiten hebben. Bluetooth Low Energy kan dus wel degelijk asset tracking op een bouwindustrie optimaliseren indien dit op een correcte manier geïmplementeerd wordt.
