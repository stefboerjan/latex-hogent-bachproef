%%=============================================================================
%% Conclusie
%%=============================================================================

\chapter{Conclusie}%
\label{ch:conclusie}

% TODO: Trek een duidelijke conclusie, in de vorm van een antwoord op de
% onderzoeksvra(a)g(en). Wat was jouw bijdrage aan het onderzoeksdomein en
% hoe biedt dit meerwaarde aan het vakgebied/doelgroep? 
% Reflecteer kritisch over het resultaat. In Engelse teksten wordt deze sectie
% ``Discussion'' genoemd. Had je deze uitkomst verwacht? Zijn er zaken die nog
% niet duidelijk zijn?
% Heeft het onderzoek geleid tot nieuwe vragen die uitnodigen tot verder 
%onderzoek?

Door middel van de vergelijkende studie en Proof of Concept is het duidelijk dat er zeker potentieel zit in het gebruik van Bluetooth Low Energy voor asset tracking op een bouwwerf. Of dit hiervoor de meest optimale of geschikte technologie zou zijn is uiteraard een andere vraag.\\

Dit is onderzocht geweest door een studie uit te voeren die de zes meest frequent voorkomende technologieën, voor asset tracking, met elkaar vergeleek. Om geen technolgien uit te sluiten is hier ook buiten de bouwsector gekeken. Hier zijn een paar criteria gebruikt die verondersteld werden belangrijk te zijn in de keuze tussen die technologieën. Al snel is er ondervonden dat barcode en QR code niet geschikt waren aangezien deze niet van op afstand gelokaliseerd kunnen worden. Tussen de andere technologieën was niet meteen zo een duidelijk verschil te zien. Voor het criterium smartphone compatibiliteit kwam RFID wat te kort aan gezien RFID niet volledig smartphone compatibel is. Terwijl BLE, UWB en GPS dit wel zijn. Indien BLE, UWB en GPS vergeleken worden met elkaar komt het aan op geven en nemen aangezien iedere technologie wel iets unieks heeft. \\

Aan de hand van een Proof of Concept is wel duidelijk gemaakt hoe matuur de BLE-standaard wel niet is en welke mogelijkheden BLE heeft out of the box. Of dit bij andere technologieën ook het geval is, is niet onderzocht geweest in deze scriptie en kan daar dus ook geen oordeel overgemaakt worden.\\

Bluetooth Low Energy zou dus wel degelijk een optimale bijdrage kunnen leveren voor asset tracking op een bouwwerf indien deze technologie op een correcte manier geïmplementeerd wordt. Wat de meest optimale manier is voor BLE te implementeren voor asset tracking doeleinden in deze sector nodigt uit tot verder onderzoek.

